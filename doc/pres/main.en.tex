% Latex template: mahmoud.s.fahmy@students.kasralainy.edu.eg
% For more details: https://www.sharelatex.com/learn/Beamer

\documentclass{beamer}          % Document class

\usepackage{babel}         % Set language
\usepackage{verbatim}
\usepackage{listings}
\usepackage{sourcecodepro}
\usepackage{xeCJK}

\mode<presentation>            % Set options
{
  \usetheme{default}          % Set theme
  \usecolortheme{default}         % Set colors
  \usefonttheme{default}          % Set font theme
  \setbeamertemplate{caption}[numbered]  % Set caption to be numbered
  \setbeamertemplate{navigation symbols}{}
}

% Uncomment this to have the outline at the beginning of each section highlighted.
%\AtBeginSection[]
%{
%  \begin{frame}{Outline}
%  \tableofcontents[currentsection]
%  \end{frame}
%}

\usepackage{graphicx}          % For including figures
\usepackage{booktabs}          % For table rules
\usepackage{hyperref}          % For cross-referencing

\title{Strongenv}  % Presentation title
\subtitle{Well-scoped binding trees in OCaml}
\author{Frédéric Bour}                % Presentation author
\institute{Tarides, Inria}          % Author affiliation
\date{July 28, 2023 \\ Nagoya}                  % Today's date

\begin{document}

% Title page
% This page includes the informations defined earlier including title, author/s, affiliation/s and the date
\begin{frame}
  \titlepage
\end{frame}

% Outline
% This page includes the outline (Table of content) of the presentation. All sections and subsections will appear in the outline by default.
\begin{frame}{Plan}
  \tableofcontents
\end{frame}

\AtBeginSection[]{
  \begin{frame}
  \vfill
  \centering
  \begin{beamercolorbox}[sep=8pt,center,shadow=true,rounded=true]{title}
    \usebeamerfont{title}\insertsectionhead\par%
  \end{beamercolorbox}
  \vfill
  \end{frame}
}

\section{Type-level tricks}

\lstset{language=caml, basicstyle=\ttfamily, basewidth=0.5em, morekeywords={module,sig,struct,val}}

\begin{frame}[fragile]{\#1/4: Singleton types for witnessing properties}

\begin{lstlisting}
module M : sig
  type 'a t
  ...
end = struct
  type 'a t = T : unit t
  ...
end
\end{lstlisting}

  \begin{itemize}
    \item External view: abstract type, nothing known about the parameter
    \item Internal view: parameter can only be \lstinline{unit}
    \item Trick: selectively reveal information about \lstinline{'a} to enforce properties
  \end{itemize}

\end{frame}

\begin{frame}[fragile]{Example: reflecting string equality at the type level}
\begin{lstlisting}
type ('a, 'b) eq = Refl : ('a, 'a) t

module Typed_string : sig
  type 'a t
  type a_t = A : 'a t -> a_t

  val inj : string -> a_t
  val prj : 'a t -> string

  val equal : 'a t -> 'b t ->
              ('a, 'b) eq option
end
\end{lstlisting}
\end{frame}

\begin{frame}[fragile]{Implementation}
\begin{lstlisting}
type 'a t = T : string -> unit t
  [@@ocaml.unboxed]

...

let equal (type a b) (A a : a t) (B b : b t)
  : (a, b) eq option =
  if String.equal a b then
    Some Refl
  else
    None
\end{lstlisting}

  \begin{itemize}
    \item User will only see existential variables for string parameters
    \item Only information that can be recovered: two existentials are equal if
          two strings are the same
  \end{itemize}
\end{frame}

\begin{frame}[fragile]{\#2/4: (Discrete) sub-typing witnesses (broken)}
  Ideally:
\begin{lstlisting}
module type SUB = sig
  type t
  type s = private t
  (* s is a sub-type of t *)
end

type ('s, 't) sub = (module SUB with type t = 't
                                 and type s = 's)
                                          ^
\end{lstlisting}
  Not working because only an abstract type can be turned into a first-class parameter.
\end{frame}

\begin{frame}[fragile]{\#2/4: (Discrete) sub-typing witnesses (fixed)}
Work-around:
\begin{lstlisting}
module type SUB = sig
  type t
  type s' = private t (* s' is a sub-type of t *)
  type s
  val eq : (s, s') eq
end

type ('s, 't) sub = (module SUB with type t = 't
                                 and type s = 's)
\end{lstlisting}

  Use an indirection and expose the equality separately.
\end{frame}

\begin{frame}[fragile]{Eliminating sub-typing witnesses}

\begin{lstlisting}
let trans_sub (type a b c)
    ((module AB) : (a, b) sub)
    ((module BC) : (b, c) sub) : (a, c) sub
  =
  let Refl = AB.eq in
  let Refl = BC.eq in
  let module Sub = struct
    type t = BC.t
    type u = AB.u
    type s = AB.s
    let eq : (s, u) eq = Refl
  end in
  ((module Sub) : (a, c) sub)
\end{lstlisting}

A bit heavy but doable while waiting for modular (ex|im)plicits :).
(See ``First-Class Subtypes'' by Yallop and Dolan)
\end{frame}

\begin{frame}[fragile]{Example: String prefixes}

\begin{lstlisting}
module Typed_string : sig
  ...
  val is_prefix : 'a t -> 'b t -> ('a, 'b) sub option
end

val refl_sub : ('a, 'a) sub

let is_prefix (type a b) (A a) (B b)
  : (a, b) sub option =
  if String.is_prefix a b then
    Some refl_sub
  else
    None
\end{lstlisting}
\end{frame}

\begin{frame}{Goal}
  We can reflect at the type level:
  \begin{itemize}
    \item an equivalence on values using equality witnesses
    \item a partial-order on values using sub-typing witnesses
  \end{itemize}

  ~ \\

  We will use this to offer a reasonably lightweight encoding of
  typing contexts ordered by inclusion.
\end{frame}

\begin{frame}[fragile]{Goal}
  \begin{lstlisting}
  val e1 : 'w1 env
  val e2 : 'w2 env
  \end{lstlisting}
  $$
  \texttt{'w1} \subseteq \texttt{'w2} \implies dom(\texttt{e1}) \subseteq dom(\texttt{e2})
  $$

  \texttt{'w} stands for ``world'', representing a set of names at the type-level.

  ~ \\

  (loosely) Following Nicolas Pouillard's PhD thesis:

  {\em ``Namely, Painless: A unifying approach to safe programming with first-order syntax with binders''}.
\end{frame}

\begin{frame}[fragile]{\#3/4: Total-ordering for heterogeneous maps}
\begin{lstlisting}
type ('a, 'b) order =
  | Lt
  | Eq : ('a, 'a) order
  | Gt

module type ORDERED = sig
  type 'a t
  val compare : 'a t -> 'b t -> ('a, 'b) order
end
\end{lstlisting}

Generalization of \lstinline{Map/Map.OrderedType} to heterogeneous situations.
\end{frame}

\begin{frame}[fragile]{\#4/4: Higher-kinded ``world'' polymorphism}

Like {\em Lightweight higher-kinded polymorphism} by Yallop and White, but restricted to {\em worlds}.
\begin{lstlisting}
module type INDEXED = sig
  type 'w t
  type p
  val pack : 'w world -> 'w t -> ('w, p) v_strong
  val unpack : 'w world -> ('w, p) v_strong -> 'w t
end
\end{lstlisting}

We have some type \lstinline{'w t} representing values with a property holding in world
\lstinline{'w}. For instance, a term well-scoped in \lstinline{'w}. \\
\pause ~ \\
A world polymorphic definition:
\only<-3>{\lstinline{forall t, w t list -> w t map} is not valid OCaml}
\only<4>{\lstinline{('w, 'p) v_strong list -> ('w, 'p) v_strong map}}
\pause
Work-around: an isomorphism $\texttt{'w t} \Leftrightarrow \texttt{('w, p) v\_strong}$. \\
\end{frame}

\begin{frame}{Summary}
  \begin{itemize}
    \item With {\em sub-typing} we can reflect inclusion of environments.
    \item With {\em heterogeneous maps}, we can represent context with multiple kind
          of binders
    \item With {\em world polymorphism}, we turn this into an efficient and
          generic toolkit enforcing ``world respecting'' usage.
  \end{itemize}
\end{frame}

\section{System-F terms}

\begin{frame}{Example: System-F AST}
  Full code available in file {\tt tests/test\_sysf.ml}
  \footnote{\url{https://gitea.lakaban.net/def/strongenv/src/branch/master/tests/test_sysf.ml}}.

\end{frame}

\begin{frame}[fragile]{System-F AST}

\begin{lstlisting}
type 'w typ =
  | Ty_var of ('w, ns_typ) ident
  | Ty_arr of 'w typ * 'w typ

  | Ty_forall : ('w1, 'w2, ns_typ) binder * 'w2 typ
                -> 'w1 typ
\end{lstlisting}

An \lstinline{'w typ} is a System-F type that is well-scoped in ``world'' \lstinline{'w}.
~\\
\pause
\textbf{Concepts}:
\begin{itemize}
  \setlength{\itemindent}{1em}
  \item[World] Type-level reflection of a set of names
  \item[Namespace] The different kinds of context binders {\small (\texttt{ns\_typ}, \texttt{ns\_term})}
  \item[Ident] \lstinline{('w, ns_typ) ident} is an identifier referring to a {\em type}
        in world \lstinline{'w}
  \item[Binder] \lstinline{('w1, 'w2, ns_typ) binder} extends world \lstinline{'w1}
        to world \lstinline{'w2}, binding an identifier to a type (dynamically, it's a pair of the identifier and the type)
\end{itemize}

\end{frame}

\begin{frame}[fragile]{System-F AST}

\begin{lstlisting}
type 'w typ =
  | Ty_var of ('w, ns_typ) ident
  | Ty_arr of 'w typ * 'w typ

  | Ty_forall : ('w1, 'w2, ns_typ) binder * 'w2 typ
                -> 'w1 typ

type 'w term =
  | Te_var of ('w, ns_term) ident
  | Te_app of 'w term * 'w term
  | Te_APP of 'w term * 'w typ

  | Te_lam : ('w1, 'w2, ns_term) binder * 'w2 term
             -> 'w1 term
  | Te_LAM : ('w1, 'w2, ns_typ) binder * 'w2 term
             -> 'w1 term
\end{lstlisting}

\end{frame}

\begin{frame}[fragile]{Boilerplate (1/2): Namespace definition}

\begin{lstlisting}[xleftmargin=-0.2in]
... and Namespace : sig

  module Type : World.INDEXED with type 'w t = 'w Syntax.typ
  module Term : World.INDEXED with type 'w t = 'w Syntax.term

type 'a t =
  | Type : Type.p t
  | Term : Term.p t

include Witness.ORDERED with type 'a t := 'a t

end
\end{lstlisting}

\begin{itemize}
  \item Construct the isomorphisms for world polymorphism and ordering for heterogeneous map.
\end{itemize}
\end{frame}

\begin{frame}[fragile]{Boilerplate (2/2): Namespace implementation}

\begin{lstlisting}[xleftmargin=-0.2in]
struct
  module Type =
    World.Indexed0(struct type 'a t = 'a Syntax.typ end)
  module Term =
    World.Indexed0(struct type 'a t = 'a Syntax.term end)

  type 'a t =
    | Type : Type.p t
    | Term : Term.p t

  let compare (type a b) (a : a t) (b : b t)
    : (a, b) Witness.order =
    match a, b with
    | Type , Type -> Eq
    | Term , Term -> Eq
    | Type , Term -> Lt
    | Term , Type -> Gt
end
\end{lstlisting}

\end{frame}

\begin{frame}[fragile]{Instantiation}

\begin{lstlisting}
...

and Context : CONTEXT with type 'a namespace = 'a Namespace.t =
  Make_context(Namespace)
\end{lstlisting}

Finally, we can instantiate the library over our syntax and namespaces. \\~\\

Note: everything is mutually recursive. The trees use idents and binders provided by the context, which uses the namespace, which associates trees to binding category, \ldots

\end{frame}

\section{Hindley-Milner inference}

\begin{frame}{Example: HM inference proof-of-concept}
  Not completely a toy:
  \begin{itemize}
    \item partial use of Remy's level to speed-up processing,
    \item typed scoping also (tries to \ldots) enforces correct use of levels.
  \end{itemize}

  ~\\

  \pause
  Full code available in files:
  \begin{itemize}
    \item {\tt tests/test\_hm\_base.ml}
          \footnote{~\href{https://gitea.lakaban.net/def/strongenv/src/branch/master/tests/test_hm_base.ml}{gitea.lakaban.net/def/strongenv/src/branch/master/tests/test\_hm\_base.ml}} \\
          Reference implementation using plain AST. \\
          Nice to see the encoding of levels first, without enforcing scoping.

     \pause
    \item {\tt tests/test\_hm\_strong.ml}
\footnote{~\href{https://gitea.lakaban.net/def/strongenv/src/branch/master/tests/test_hm_strong.ml}{gitea.lakaban.net/def/strongenv/src/branch/master/tests/test\_hm\_strong.ml}} \\
          Well-scoped implementation using Strongenv.
  \end{itemize}

\end{frame}

\begin{frame}[fragile]{Explicit level manipulation}

\begin{lstlisting}
type level
type ty_var

val initial_level : unit -> level

val new_var : level -> ty_var

val begin_level : level -> level
val end_level : level -> ty_var list (*generalized*)
\end{lstlisting}

\end{frame}

\begin{frame}[fragile]{Level representation}

\begin{lstlisting}
type level = { mutable level_repr: level_repr }

and level_repr =
  | Fresh of {
      gensym: int ref;
      depth: int;
      mutable variables: ty_var list;
    }
  | Generalized of ty_var list;

and ty_var = {
  id: int;
  mutable level: level;
  mutable repr: typ;
}
\end{lstlisting}

\end{frame}

\begin{frame}[fragile]{Level representation: scoped version}

\begin{lstlisting}
type 'w level = { mutable level_repr : 'w level_repr; }

and 'w level_repr =
  | Fresh of {
      gensym: int ref;
      world: 'w World.t;
      mutable variables : 'w ty_var list;
    }
  | Generalized of 'w ty_var list

and 'w ty_var = {
  id: int;
  mutable level: 'w level;
  mutable repr: 'w typ;
}
\end{lstlisting}

\end{frame}

\begin{frame}[fragile]{Scoped version}

  Two namespaces:
  \begin{itemize}
    \setlength{\itemindent}{3em}
    \item[ns\_value] to give types to program variables, with proper scoping
    \item[ns\_level] to represent a set of type variables
  \end{itemize}
  \pause ~\\
  During inference, the set of type variables is being discovered and can grow: not suitable for the static structure of the ``well-scoped AST''. \\
  However, level placement is fixed: store levels in the AST.

\pause ~\\
\begin{lstlisting}
type 'w term =
  ...
  | Te_let : {
      level: ('w, 'wl, ns_level) binder;
      bound: 'wl term;
      binder: ('w, 'wb, ns_value) binder;
      body: 'wb term;
    } -> 'w term_desc
\end{lstlisting}

\end{frame}

\begin{frame}[fragile]{Interaction of pure and impure code}

  We are mixing pure and impure constructions:
  \begin{itemize}
    \item immutable environments and ASTs
    \item growing levels, moving type variables
  \end{itemize}

  \pause ~\

  Changes and invariants are monotonous:
  \begin{itemize}
    \item variables can only move toward the root level
    \item levels can only go from fresh to generalized
    \item once generalized, they are effectively ``pure''
    \item[$\Rightarrow$] internal error if broken (assertion fails)
  \end{itemize}

\end{frame}

\begin{frame}[fragile]{Subtle case: opt-out of safety}

  After generalizing a type \lstinline{'w2 Syntax.typ}, we know we can use it in the outer level \lstinline{'w1 level}.

  However from the point of view of the type system, \lstinline{'w2} and \lstinline{'w1} are incompatible.

We can implement an unsafe primitive for this case:
\begin{lstlisting}
val escape_typ :
  ('w1, 'w2, Syntax.ns_value) Context.binder ->
  'w2 Syntax.typ -> 'w1 Syntax.typ
\end{lstlisting}

This is a subtle code where types could escape their scope in a normal type-checker. It is caught by our encoding: we have to explicitly opt-out of safety. Now we know where dangerous things happen.

\end{frame}

\begin{frame}{That's all, thanks!}

  ありがとうございます。

\end{frame}

\end{document}
